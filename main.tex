\documentclass[conference]{IEEEtran}

\usepackage{algorithmicx}
\usepackage[noend]{algpseudocode}
\usepackage[ruled,vlined]{algorithm2e}

\algblock{ParFor}{EndParFor}
% customising the new block
\algnewcommand\algorithmicparfor{\textbf{ParFor}}
\algnewcommand\algorithmicpardo{\textbf{do}}
\algnewcommand\algorithmicendparfor{\textbf{end\ ParFor}}
\algrenewtext{ParFor}[1]{\algorithmicparfor\ #1\ \algorithmicpardo}
\algrenewtext{EndParFor}{\algorithmicendparfor}

% for numbered citations
\usepackage{cite}
% for figures based on pdfs
\usepackage[pdftex]{graphicx}
 
 % Ben's packages
\usepackage{pgfplots}
\pgfplotsset{compat=1.13}
\usepackage{times}
\usepackage{listings}
\graphicspath{ {images/} }


\begin{document}

\title{Enabling Usage of Parallelism In Python}
\author{
\IEEEauthorblockA{Benjamin James Gaska}
\IEEEauthorblockA{Computer Science\\
University of Arizona\\
Tucson, Arizona\\
Email: bengaska@email.arizona.edu}}

\maketitle

\begin{abstract}
TEST
\end{abstract}

\section{Introduction}




\section{Motivation}



\subsection{Case Study: Parallelizing Orbital Code}



\section{Survey}

\section{Implementation}

\section{Results}

The ParPython implementation strives for simplicity, ease of use, and minimal difference between the serial and parallel code. 
To this end, 100 benchmarks were chosen and and an attempt was made to parallelize it using only the ParPython loop parallelism model.

The benchmarks were sampled from a larger set of benchmarks made up of the following:
\begin{itemize}
   \item Python Performance Suite\cite{pyPerformance}, a collection of benchmarks for comparison of Python implementations
   \item Numpy Benchmark Suite\cite{numpyPerformance}, the official test suite for Numpy
   \item Scikit-Learn Benchmarks \cite{scikit-learn}, the official benchmarks for the popular Python machine learning library
   \item Natural Language Toolkit (NLTK) \cite{bird_2016}, a popular open-source codebase for natural language processing in Python.
\end{itemize}
These codebases were chosen because they represent a variety of code that include both implementation entirely in Python, as well as implementations that make significant calls to external C libraries. 
They also represent comment tools used for dataset analysis in Python, and thus one of the major use cases of Python in scientific research.



The benchmarks were then classified into one of four possible categories:
\begin{enumerate}
   \item No modification required beyond inserting ParFor,
   \item some refactoring was required to implement using the ParFor, 
   \item a large amount of refactoring of code logic was required to use ParFar,
   \item or code could not be parallelized using ParFor.
\end{enumerate}



\section{Related Work}

\section{Conclusions}


\bibliographystyle{IEEEtran}
\bibliography{Thesis}
\end{document}
